% These codes creates two commands and an environment making it easier to insert images in your report.
% !! Please be awared that these commands assume that your figures are converted into eps format and placed in eps/

% \cvimg{filname}{width_percentage}{title}
%%
%%       where filname - the filename of figure, "fig_1" refering to "eps/fig_1.eps", please do not contain any dot (".") in the filename
%%                      you can also refer to it, using \ref{fig:filename}
%%            width_percentage - the width of figure, it's a number in [0, 1.0] refering to the percentage of width of page
%%            title - the caption of figure

% \begin{cvimggroup}{group_label}{Group Title}
%    \cvimgitem{fig_1}{0.3}{Image Title 1}
%    \cvimgitem{fig_2}{0.3}{Image Title 2}
%    \cvimgitem{fig_3}{0.3}{Image Title 3}
% \end{cvimggroup}
%%
%% cvimggroup allows you put several figures in the same row
%% cvimgitem works in a similar way as cvimg does
%%    the width argument (the 2nd one) of cvimgitem refers to the percentage of width of the page
%%    images may be seprated into different rows if width is too large
%%    if 0.33 does not work, please try 0.3 or 0.25
%%    (if you have a good way to solve it and you'd like to share it with us, please feel free to submit a Pull Request)

% Please star this repo if it helps :P


\newcommand{\cvimg}[3]{
    \begin{figure}[H]
        \centering
        \includegraphics[width=#2\textwidth]{eps/#1.eps}
        \caption{#3}
        \label{fig:#1}
    \end{figure}
}

\newenvironment{cvimggroup}[2]{
    \begin{figure}[htbp]
    \def\cvimggrouplabels{\caption{#2}\label{#1}}
    \centering
}{
    \cvimggrouplabels
    \end{figure}
}


\newcommand{\cvimgitem}[3]{
    \begin{subfigure}[b]{#2\textwidth}
        \centering
        \includegraphics[width=\textwidth]{eps/#1.eps}
        \caption{#3}
        \label{fig:#1}
    \end{subfigure}
}
